\documentclass[fontsize=11pt]{article}
\usepackage[english]{babel}
\usepackage[utf8]{inputenc}
\usepackage[T1]{fontenc}
%\usepackage{lmodern}
\usepackage[protrusion=true,expansion=true]{microtype}
\usepackage[svgnames]{xcolor}  % Colours by their 'svgnames'
\usepackage[margin=0.75in]{geometry}
  \textheight=700px
\usepackage{url}
\usepackage[normalem]{ulem}

\usepackage{wasysym,marvosym,fontawesome}

\usepackage{newtxtext}
\usepackage{lmodern} % Allow arbitrary font sizes
\usepackage{textcomp}

%% Define a new 'modern' style for the url package that will use a smaller font.
\makeatletter
\def\url@modernstyle{
  \@ifundefined{selectfont}{\def\UrlFont{\sf}}{\def\UrlFont{}}}
\makeatother
\urlstyle{modern} %% And use the newly defined style.

\frenchspacing              % Better looking spacings after periods
\pagestyle{empty}           % No pagenumbers/headers/footers

%\renewcommand{\familydefault}{\sfdefault}

%%% Custom sectioning (sectsty package)
%%% ------------------------------------------------------------
\usepackage{sectsty}

\sectionfont{                 % Change font of \section command
  \large\usefont{OT1}{phv}{b}{n}%   % bch-b-n: CharterBT-Bold font
  \sectionrule{0pt}{0pt}{-5pt}{1pt}}

%%% Macros
%%% ------------------------------------------------------------
\newlength{\spacebox}
\settowidth{\spacebox}{8888888888}      % Box to align text
\newcommand{\sepspace}{\vspace*{1em}}   % Vertical space macro

\newcommand{\MyName}[1]{ % Name
    \LARGE \usefont{OT1}{phv}{b}{n}
    %\hfill
    \begin{center}
        #1
    \end{center}
    %\par
    \normalsize \normalfont}

\newcommand{\MySlogan}[1]{ % Slogan (optional)
    \large \usefont{OT1}{phv}{m}{n}\hfill \textit{#1}
    \par \normalsize \normalfont}

%\newcommand{\NewPart}[1]{\section*{\uppercase{#1}}}
\newcommand{\NewPart}[1]{\section*{#1}}

\newcommand{\PersonalEntry}[2]{
    \noindent\hangindent=2em\hangafter=0 % Indentation
    \parbox{\spacebox}{                  % Box to align text
    \textit{#1}}                      % Entry name (birth, address, etc.)
    \hspace{1.5em} #2 \par}              % Entry value

\newcommand{\SkillsEntry}[2]{                % Same as \PersonalEntry
    \noindent\hangindent=2em\hangafter=0 % Indentation
    \parbox{\spacebox}{                  % Box to align text
    \textit{#1}}                    % Entry name (birth, address, etc.)
    \hspace{1.5em} #2 \par}              % Entry value

\newcommand{\AwardsEntry}[2]{                % Same as \PersonalEntry
    \noindent\hangindent=2em\hangafter=0 % Indentation
    \parbox{\spacebox}{                  % Box to align text
    \textit{#1}}                    % Entry name (birth, address, etc.)
    \hspace{1.5em} #2 \par}              % Entry value

\newcommand{\EducationEntry}[4]{
    \noindent \textbf{#1}
    \textit{#3}
    \hfill      % Study
    \colorbox{White}{
      \parbox{9em}{
      \hfill\color{Black}#2}} \par  % Duration
    %\noindent \textit{#3} \par        % School
    \noindent\hangindent=2em\hangafter=0 \small #4 % Description
    \normalsize \par}

\newcommand{\WorkEntry}[4]{       % Same as \EducationEntry
    \noindent
    \textbf{#1}
    \textit{#3} %\par        % Company
    \hfill      % Jobname
    \colorbox{White}{%
      \parbox{9em}{%
      \hfill\color{Black}#2}} \par   % Duration
        %\noindent \textit{#3} \par        % Company
    \noindent\hangindent=2em\hangafter=0 \small #4 % Description
    \normalsize \par}

\newcommand{\ProjectEntry}[4]{         % Similar to \EducationEntry
    \noindent \textbf{#1} \noindent {#2} {#3} \par
    \noindent \small #4 % Description
    \normalsize \par}

\newcommand{\AwardEntry}[4]{         % Similar to \EducationEntry
    \noindent \textbf{#1} \noindent \textit{#2} \hfill {#3} \par
    \noindent \small #4 % Description
    \normalsize \par}
    \begin{document}
    
\MyName{Aditya Krishna Menon}
{\par
\begin{center}
    % {\footnotesize{\footnotesize\phone}\ +61 439 761 969}
    % $\ $
    % {\footnotesize\tt{\footnotesize\Letter}\ aditya.menon@anu.edu.au}
    % $\ $
    % {\footnotesize\tt{\footnotesize\faGlobe}\ \url{http://users.cecs.anu.edu.au/~akmenon/}} \\    
    % {\footnotesize\tt{\footnotesize\faGraduationCap}\ \url{http://scholar.google.com.au/citations?user=li4mEfcAAAAJ}}
    {\small{\small\faMapMarker}\ New York, NY}
    %$\ $
    %{\small{\small\phone}\ +1 646 262 0466}
    $\ $
    {\small\tt{\small\Letter}\ adityakmenon@google.com}
\end{center}
\par}

%\bigskip
%{\small \hfill }


%%% Work experience
%%% ------------------------------------------------------------
\NewPart{Experience}{}

\WorkEntry
{Staff Research Scientist}
{Nov 2020 -- Present}
{Google}{\vspace{-\baselineskip}}
\WorkEntry
{Senior Research Scientist}
{Sep 2018 -- Oct 2020}
{Google}
{%
\begin{itemize} \itemsep -4pt
    \item Working on new techniques for large-scale classification and language models
    \item Managing a team of $8$ research scientists
\end{itemize}
}

\WorkEntry
{Honorary Senior Lecturer}
{Jul 2018 -- Aug 2021}
{Australian National University}{\vspace{-\baselineskip}}
\WorkEntry
{Fellow}
{Jan 2018 -- Jul 2018}
{Australian National University}
% \WorkEntry
% {Honorary Lecturer}
% {May 2016 -- Jan 2018}
% {Australian National University}{\vspace{-\baselineskip}}
% \WorkEntry
% {Adjunct Research Fellow}
% {May 2013 -- May 2016}
% {Australian National University}
{%
\begin{itemize} \itemsep -1pt
    \item Analysing different means of imposing ``fairness'' constraints on classifiers, and their resulting tradeoffs 

    \item Designing algorithms to predict popularity of content on social media, e.g., videos on YouTube

    \item Performing academic duties, including co-supervision of two PhD students
\end{itemize}
}

\WorkEntry
{Senior Research Scientist}
{Jul 2016 -- Dec 2017}
{CSIRO Data61}
{
\begin{itemize} \itemsep -1pt
        \item
            Published research on theoretical \& applied machine learning topics, e.g., Bregman divergences, point processes, recommender systems

        \item
            Led machine learning for industrial projects on {transport congestion management} and {border security}

        \item
            Performed academic duties at the Australian National University, including co-supervision of two PhD students
\end{itemize}
}

\WorkEntry
{Researcher}
{May 2013 -- Jun 2016}
{National ICT Australia (NICTA)}
{
\begin{itemize} \itemsep -1pt
        \item
            Published research on theoretical \& applied machine learning topics, e.g., bipartite ranking, label noise, recommender systems

        \item
            Involved in machine learning for industrial projects on {solar energy forecasting} and urban mobility

        \item
            Performed academic duties at the Australian National University, including co-supervision of two PhD students, and lecturing
\end{itemize}
}

\WorkEntry
{Data Scientist Intern}
{Jun 2012 -- Sep 2012}
{LinkedIn}
{
\begin{itemize} \itemsep -1pt
        \item Worked on end-to-end system for using machine learning to automate search log analysis
\end{itemize}
}

\WorkEntry
{Research Intern}
{Jun 2011 -- Sep 2011}
{Microsoft Research New England}
{
\begin{itemize} \itemsep -1pt
        \item Worked on using machine learning to automatically infer user's intent for repetitive text processing tasks
        %\item Culminated in an ICML paper.
\end{itemize}
}

\WorkEntry
{Research Intern}
{Jun 2010 -- Sep 2010}
{Yahoo! Labs Bangalore}
{
\begin{itemize} \itemsep -1pt
        \item Worked on estimating the clickthrough rate of ads on webpages using collaborative filtering
        %\item Culminated in a KDD paper.    
\end{itemize}
}


%%% Education
%%% ------------------------------------------------------------
\NewPart{Education}{}

\EducationEntry
{PhD in Computer Science}
%{Sep 2007 -- Mar 2013}
{Mar 2013}
{University of California, San Diego}
{
%\hspace{-2pt}Degree conferred March 23rd, 2013 \\
\hspace{-2pt}\emph{Thesis title}: Latent feature models for dyadic prediction \\
\emph{Supervisor}: Charles Elkan
}

\vskip0.25\baselineskip

% \EducationEntry
% {CPhil in Computer Science}
% {Sep 2007 -- Jun 2011}
% {University of California, San Diego}
% {
% }

% \vskip-0.5\baselineskip

% \EducationEntry
% {MS in Computer Science}
% {Sep 2007 -- Jun 2009}
% {University of California, San Diego}
% {
% }

% \vskip-0.5\baselineskip

\EducationEntry
{BSc (Advanced) Honours in Computer Science}
%{Mar 2003 -- Nov 2006}
%{Mar 2003 -- May 2007}
{May 2007}
{The University of Sydney}
{
%\hspace{-2pt}Degree conferred May 25th, 2007 \\
\hspace{-2pt}First Class Honours, University Medal, \& Allan Bromley Prize for best thesis in Computer Science \\
\emph{Thesis title}: Random projections and applications to dimensionality reduction \\
\emph{Supervisor}: Sanjay Chawla
}


%%% Awards
%%% ------------------------------------------------------------
\NewPart{Awards}{}

% \AwardEntry
% {Outstanding Reviewer}
% {International Conference on Machine Learning}
% {2018}
% {Awarded for ...}

\AwardEntry
{Best Technical Contribution Award}
{Conference on Fairness, Accountability, and Transparency}
{2018}
%{Awarded for ...}

\AwardEntry
{Research Excellence Award}
{Intelligent Transport Systems Australia}
{2014 -- 2015}
    {Awarded to Advanced Data Analytics in Transport team}

\AwardEntry{Student Travel Award}
{International Conference on Data Mining}
{2010}

\AwardEntry{Jacobs Fellowship}
{University of California, San Diego}
{2007 -- 2009}
%{Prestigious fellowship awarded to incoming PhD students}

\AwardEntry{University Medal}
{The University of Sydney}
{2007}

\AwardEntry{Allan Bromley Prize}
{The University of Sydney}
{2007}
%{Awarded for best honours thesis}

\AwardEntry{Continuing Undergraduate Scholarship}
{The University of Sydney}
{2004 -- 2006}

\AwardEntry{Talented Student Program}
{The University of Sydney}
{2003 -- 2005}

% \AwardEntry{Farrand Science Scholarship}
% {The University of Sydney}
% {2003}


%%% Skills
%%% ------------------------------------------------------------
\NewPart{Research Interests}{}

\noindent
Weakly-supervised learning (e.g., learning from label noise, positive and unlabelled learning) %\\

\noindent
Classification with real-world constraints (e.g., class imbalance, fairness) %\\

\noindent
Matrix factorisation \& applications (e.g., collaborative filtering, link prediction) %\\



%%%
\NewPart{Selected Academic Research Publications}{}

    \ProjectEntry
    {The cost of fairness in binary classification.}
    {Aditya Krishna Menon and Robert C. Williamson.}
    {In \emph{Conference on Fairness, Accountability, and Transparency (\textbf{FAT})}, 2018. 
    \uline{{Best Technical Contribution.}}
    }
    {{
    \color{gray}
    \indent
    Explicates how the inherent tradeoff between accuracy and fairness depends on the alignment of the distributions for each task.
    To achieve this, we show that %two popular measures of fairness are intimately connected to cost-sensitive losses, and that
    the Bayes-optimal fairness-aware classifiers involve \emph{instance-dependent} thresholding of the class-probability.
    }}

\vskip0.5\baselineskip

    % \ProjectEntry
    % {Making deep neural networks robust to label noise: a loss correction approach.}
    % {Giorgio Patrini, Alessandro Rozza, Aditya Krishna Menon, Richard Nock, Lizhen Qu.}
    % {In \emph{Computer Vision and Pattern Recognition (\textbf{CVPR})}, 2017.
    % \uline{{36 citations.}}}
    % {{
    % \color{gray} 
    % \indent
    % Shows that when the input labels to a deep network are subject to random noise, we can estimate the noise rate and subsequently re-weight our loss function to account for uncertainty in the provided labels. This yields a simple, architecture-independent robustification procedure.
    % }}

% \vskip0.5\baselineskip

%   \ProjectEntry
%   {Bipartite ranking: a risk-theoretic perspective.}
%   {Aditya Krishna Menon and and Robert C. Williamson.}
%   {In \emph{Journal of Machine Learning Research (\textbf{JMLR})}, 2016.}
%   {{
%   \color{gray}
%   \indent
%   Analyses the bipartite ranking problem in terms of its underlying statistical risk.
%   This is used to show that certain surrogates to the AUC will recover the correct (Bayes-optimal) solution, and that one can design surrogate losses to emphasise accuracy at the head of the ranked list.
%   }}

% \

%     \ProjectEntry
%     {A scaled Bregman theorem with applications.}
%     {Richard Nock, Aditya Krishna Menon and Cheng Soon Ong.}
%     {In \emph{Advances in Neural Processing Systems (\textbf{NIPS})}, 2016.}
%     {{
%     \color{gray}
%     \indent
%     Establishes a formal reduction between the density ratio and class-probability estimation problems. This is done via a novel identity for Bregman divergences, and justifies using methods like logistic regression to estimate covariate shift levels between train and test sets.
%     }}

% \

%     \ProjectEntry
%     {Linking losses for density ratio and class-probability estimation.}
%     {Aditya Krishna Menon and Cheng Soon Ong.}
%     {In \emph{International Conference on Machine Learning (\textbf{ICML})}, 2016.}
%     {{
%     \color{gray}
%     \indent
%     Establishes a formal reduction between the density ratio and class-probability estimation problems. This is done via a novel identity for Bregman divergences, and justifies using methods like logistic regression to estimate covariate shift levels between train and test sets.
%     }}

% \vskip0.5\baselineskip

%   \ProjectEntry{Learning with symmetric label noise: The importance of being unhinged.}
% Brendan van Rooyen, Aditya Krishna Menon and and Robert C. Williamson.
% In \emph{Advances in Neural Information Processing Systems (\textbf{NIPS})}, 2015.
%   %{\color{blue}{14 citations}}
%   \\
%   {\color{gray} Shows that a simple convex loss is provably robust to symmetric label noise, and established connections of the same to highly regularised SVMs.
%   }

    \ProjectEntry
    {Learning from corrupted binary labels via class-probability estimation.}
    {Aditya Krishna Menon, Brendan van Rooyen, Cheng Soon Ong and Robert C. Williamson.}
    {In \emph{International Conference on Machine Learning (\textbf{ICML})}, 2015.
    %\uline{{41 citations.}}
    }
    {{\color{gray}
    \indent
    Shows that when binary labels are corrupted with noise, the noise rate can be inferred from class-probability estimates, with no access to clean samples. This is done by relating the clean and noisy class-probabilities, generalising existing results for special cases.
    }}

\vskip0.5\baselineskip

    \ProjectEntry
    {AutoRec: autoencoders meet collaborative filtering.}
    {Suvash Sedhain, Aditya Krishna Menon, Scott Sanner, Lexing Xie.}
    {In \emph{International Conference on World Wide Web (\textbf{WWW})}, 2015.
    %\uline{{113 citations.}}
    }
    {{
    \color{gray}
    \indent    
    Introduces a new means of predicting user ratings for content, wherein a non-linear autoencoder is applied to each row of the rating matrix. This simple approach was shown to outperform matrix factorisation, which has long been the \emph{de-facto} approach to collaborative filtering.
    }}

\vskip0.5\baselineskip

    % \ProjectEntry{Bayes-optimal scorers for bipartite ranking.}
    % {Aditya Krishna Menon and Robert C. Williamson.}
    % {In \emph{Conference on Learning Theory (\textbf{COLT})}, 2014.}
    % {{
    % \color{gray}
    % \indent
    % Explicates a subtlety
    % in using surrogate losses for bipartite ranking,
    % owing to an implicit restriction on the function class.
    % Establishes that for a broad class of surrogates,
    % we nonetheless have consistency and surrogate regret bounds via
    % a reduction to pairwise classification.
    % }}

    \ProjectEntry
    {Response prediction using collaborative filtering with hierarchies and side-information.}
    {Aditya Krishna Menon, Krishna-Prasad Chitrapura, Sachin Garg, Deepak Agarwal, and Nagaraj Kota.}
    {In \emph{Knowledge Discovery \& Data Mining (\textbf{KDD})}, 2011.
    %\uline{{110 citations.}}
    }
    {{
    \color{gray}
    \indent
    Re-casts the problem of predicting clickthrough rates of ads as one of ``recommending'' ads to webpages, thereby allowing the use of techniques developed for collaborative filtering.
    This approach significantly improves performance over existing approaches.
    }}    

\vskip0.5\baselineskip

    \ProjectEntry
    {Link prediction via matrix factorization.}
    {Aditya Krishna Menon and Charles Elkan.}
    {In \emph{Machine Learning and Knowledge Discovery in Databases}, 2011.
    %\uline{{342 citations.}}
    }
    {{
    \color{gray}
    \indent
    Proposes a novel means of predicting link formation in graphs, by observing the similarity between this problem and collaborative filtering.
    }}


\vskip0.5\baselineskip


%%% Skills
%%% ------------------------------------------------------------
\NewPart{Selected Industrial Research Projects}{}

\WorkEntry
{Inverse problems for road traffic}
{Aug 2013 - Dec 2014}
{NICTA and Transport for NSW}
{
\begin{itemize} 
    \itemsep-0.1\baselineskip
    \item Worked with a diverse team including transportation scientists and research engineers             
    \item Developed learning algorithms to solve an inverse problem central to transport science
    \item Implemented algorithms in {\tt python} and {\tt MATLAB}, and engaged with engineers to integrate into live demos
    \item Work culminated in team receiving 2014 \& 2015 Intelligent Transport Systems Research award, and publication in top transport journal
\end{itemize}
}

\WorkEntry
{Loss functions for solar energy forecasting}
{Jun 2013 - Jul 2016}
{NICTA and Australian Renewable Energy Agency}
{
\begin{itemize}
    \itemsep-0.1\baselineskip
    \item Worked on designing performance measures for forecasting of energy output from distributed solar panels
    \item Demonstrated viability of measures from class-imbalance literature to measure detection rate of ``ramp'' events
    \item Engaged with and presented findings to stakeholders in industry and government
    %\item work culminated in publication in leading photovoltaic conference
    \item Project was positively received by sponsoring government agency, and awarded additional funds to continue research
\end{itemize}
}

\WorkEntry
{Anomaly detection for border protection}
{Jan 2017 - Mar 2017}
{CSIRO Data61 and Unisys}
{
\begin{itemize}
    \itemsep-0.1\baselineskip
    \item Worked to enhance Unisys' border risk-assessment platform
    \item Designed machine learning algorithms for detecting anomalies in cargo and passenger data
    \item Set overall modelling and implementation strategy, and oversaw work of research engineer
    \item Work culminated in continued engagement with client, and favourable media coverage
\end{itemize}
}


% %%% Skills
% %%% ------------------------------------------------------------
\NewPart{Teaching Experience}{}

\WorkEntry
{Guest Lecturer}
{October 2023}
{National University of Singapore}
{
    DSA4262 Sense-Making Case Analysis: Health and Medicine
}

\WorkEntry
{Guest Lecturer}
{October 2021}
{University of Sydney}
{
    COMP5328: Advanced Machine Learning
}

\WorkEntry
{Lecturer}
{Jul -- Aug 2013 -- 2016}
{Australian National University}
{
    COMP2610/COMP6261: Information Theory
}

\WorkEntry
{Guest Lecturer}
{Jul -- Aug 2013 -- 2016}
{Australian National University}
{
    COMP2610/COMP6261: Information Theory
}

\WorkEntry
{Teaching assistant}
{Jan -- Mar 2009 -- 2012}
{University of California, San Diego}
{
    COMP101: Algorithms; COMP250A: Probabilistic Reasoning and Decision-Making; COMP250B: Learning
}

%%% Skills
%%% ------------------------------------------------------------
\NewPart{Programming Languages}{}

\emph{Proficient}: {{\tt python} + scientific toolkit ({\tt numpy}, {\tt scipy}, {\tt sklearn})},
{{\tt MATLAB}}

\noindent\emph{Familiar}: \,\ {\tt C}, {\tt C++}, {\tt Java},
{{\tt TensorFlow}},
{{\tt JAX}}


%%% Skills
%%% ------------------------------------------------------------
\NewPart{Professional Service}{}

\noindent
Program Committee member for 
ICML 2012--2019; 
NIPS 2011, 2015--2019; 
AAAI 2012, 2015--2019; 
AISTATS 2016, 2018;
IJCAI 2016, 2018; 
KDD 2011--2012; 
CIKM 2013; 
ACML 2014--2015;
ACL 2024;
COLM 2024

\medskip

\noindent
Senior Program Committee (Area Chair) member for 
IJCAI 2017, 
AAAI 2019, 
ICML 2021--2024, 
NeurIPS 2020--2024, 
ICLR 2020--2024, 
ACML 2020--2022

\medskip

\noindent
Action Editor for Transactions on Machine Learning Research

\medskip

\noindent
Reviewer for Journal of Machine Learning Research, Machine Learning, Data Mining and Knowledge Discovery, Biometrika, Transactions on Pattern Analysis and Machine Intelligence

\medskip

\noindent
Recognised as Outstanding Reviewer for ICML 2018, 2019 and NeurIPS 2018

\medskip

\noindent
Recognised as Outstanding Senior Program Committee member for AAAI 2019

%%%
\NewPart{Full List of Academic Research Publications}{}

\ProjectEntry
{Language model cascades: token-level uncertainty and beyond.}
{Neha Gupta, Harikrishna Narasimhan, Wittawat Jitkrittum, Ankit Singh Rawat, Aditya Krishna Menon, and Sanjiv Kumar.}
{In International Conference on Learning Representations (ICLR), 2024.}
{{\vspace{-0.5\baselineskip}}}

\ProjectEntry
{Learning to reject meets long-tail learning.}
{Harikrishna Narasimhan, Aditya Krishna Menon, Wittawat Jitkrittum, Neha Gupta, and Sanjiv Kumar}
{In International Conference on Learning Representations (ICLR), 2024.}
{{\vspace{-0.5\baselineskip}}}

\ProjectEntry
{Plugin estimators for selective classification with out-of-distribution detection.}
{Harikrishna Narasimhan, Aditya Krishna Menon, Wittawat Jitkrittum, and Sanjiv Kumar.}
{In International Conference on Learning Representations (ICLR), 2024.}
{{\vspace{-0.5\baselineskip}}}

\ProjectEntry
{DistillSpec: improving speculative decoding via knowledge distillation.}
{Yongchao Zhou, Kaifeng Lyu, Ankit Singh Rawat, Aditya Krishna Menon, Afshin Rostamizadeh, Sanjiv Kumar, Jean-François Kagy, and Rishabh Agarwal.}
{In International Conference on Learning Representations (ICLR), 2024.}
{{\vspace{-0.5\baselineskip}}}

\ProjectEntry
{Think before you speak: training language models with pause tokens.}
{Sachin Goyal, Ziwei Ji, Ankit Singh Rawat, Aditya Krishna Menon, Sanjiv Kumar, and Vaishnavh Nagarajan.}
{In International Conference on Learning Representations (ICLR), 2024.}
{{\vspace{-0.5\baselineskip}}}

\ProjectEntry
{The importance of feature preprocessing for differentially private linear optimization.}
{Ziteng Sun, Ananda Theertha Suresh, and Aditya Krishna Menon.}
{In International Conference on Learning Representations (ICLR), 2024.}
{{\vspace{-0.5\baselineskip}}}

\ProjectEntry
{What do larger image classifiers memorise?}
{Michal Lukasik, Vaishnavh Nagarajan, Ankit Singh Rawat, Aditya Krishna Menon, and Sanjiv Kumar.}
{In Transactions of Machine Learning Research (TMLR), 2024.}
{{\vspace{-0.5\baselineskip}}}

\ProjectEntry
{When does confidence-based cascade deferral suffice?}
{Wittawat Jitkrittum, Neha Gupta, Aditya Krishna Menon, Harikrishna Narasimhan, Ankit Singh Rawat, and Sanjiv Kumar.}
{In Advances in Neural Information Processing Systems (NeurIPS), 2023.}
{{\vspace{-0.5\baselineskip}}}

\ProjectEntry
{On student-teacher deviations in distillation: does it pay to disobey?}
{Vaishnavh Nagarajan, Aditya Krishna Menon, Srinadh Bhojanapalli, Hossein Mobahi, and Sanjiv Kumar.}
{In Advances in Neural Information Processing Systems (NeurIPS), 2023.}
{{\vspace{-0.5\baselineskip}}}

\ProjectEntry
{ResMem: Learn what you can and memorize the rest.}
{Zitong Yang, Michal Lukasik, Vaishnavh Nagarajan, Zonglin Li, Ankit Singh Rawat, Manzil Zaheer, Aditya Krishna Menon, and Sanjiv Kumar.}
{In Advances in Neural Information Processing Systems (NeurIPS), 2023.}
{{\vspace{-0.5\baselineskip}}}

\ProjectEntry
{Robust distillation for worst-class performance: on the interplay between teacher and student objectives.}
{Serena Wang, Harikrishna Narasimhan, Yichen Zhou, Sara Hooker, Michal Lukasik, and Aditya Krishna Menon.}
{In Uncertainty in Artificial Intelligence (UAI), 2023.}
{{\vspace{-0.5\baselineskip}}}

\ProjectEntry
{Supervision complexity and its role in knowledge distillation.}
{Hrayr Harutyunyan, Ankit Singh Rawat, Aditya Krishna Menon, Seungyeon Kim, and Sanjiv Kumar.}
{In International Conference on Learning Representations (ICLR), 2023.}
{{\vspace{-0.5\baselineskip}}}

\ProjectEntry
{Post-hoc estimators for learning to defer to an expert.}
{Harikrishna Narasimhan, Wittawat Jitkrittum, Aditya Krishna Menon, Ankit Singh Rawat, and Sanjiv Kumar.}
{In Advances in Neural Information Processing Systems (NeurIPS), 2022.}
{{\vspace{-0.5\baselineskip}}}

\ProjectEntry
{Teacher's pet: understanding and mitigating biases in distillation.}
{Michal Lukasik, Srinadh Bhojanapalli, Aditya Krishna Menon, and Sanjiv Kumar.}
{In Transactions of Machine Learning Research (TMLR), 2022.}
{{\vspace{-0.5\baselineskip}}}

\ProjectEntry
{Interval-censored Hawkes processes.}
{Marian-Andrei Rizoiu, Alexander Soen, Shidi Li, Pio Calderon, Leanne J. Dong, Aditya Krishna Menon, and Lexing Xie.}
{In Journal of Machine Learning Research (JMLR), 2022.}
{{\vspace{-0.5\baselineskip}}}

\ProjectEntry
{In defense of dual-encoders for neural ranking.}
{Aditya Krishna Menon, Sadeep Jayasumana, Ankit Singh Rawat, Seungyeon Kim, Sashank J. Reddi, and Sanjiv Kumar.}
{In International Conference on Machine Learning (ICML), 2022.}
{{\vspace{-0.5\baselineskip}}}

\ProjectEntry
{Training over-parameterized models with non-decomposable objectives.}
{Harikrishna Narasimhan and Aditya Krishna Menon.}
{In Advances in Neural Information Processing Systems (NeurIPS), 2021.}
{{\vspace{-0.5\baselineskip}}}

\ProjectEntry
{Disentangling sampling and labeling bias for learning in large-output spaces.}
{Ankit Singh Rawat, Aditya Krishna Menon, Wittawat Jitkrittum, Sadeep Jayasumana, Felix X. Yu, Sashank Reddi, and Sanjiv Kumar.}
{In International Conference on Machine Learning (ICML), 2021.}
{{\vspace{-0.5\baselineskip}}}

\ProjectEntry
{A statistical perspective on distillation.}
{Aditya Krishna Menon, Ankit Singh Rawat, Sashank J. Reddi, Seungyeon Kim, and Sanjiv Kumar.}
{In International Conference on Machine Learning (ICML), 2021.}
{{\vspace{-0.5\baselineskip}}}

\ProjectEntry
{RankDistil: knowledge distillation for ranking.}
{Sashank Reddi, Rama Kumar Pasumarthi, Aditya Krishna Menon, Ankit Singh Rawat, Felix Yu, Seungyeon Kim, Andreas Veit, and Sanjiv Kumar.}
{In Artificial Intelligence and Statistics (AISTATS), 2021.}
{{\vspace{-0.5\baselineskip}}}

\ProjectEntry
{Coping with label shift via distributionally robust optimisation.}
{Jingzhao Zhang, Aditya Krishna Menon, Andreas Veit, Srinadh Bhojanapalli, Sanjiv Kumar, and Suvrit Sra.}
{In International Conference on Learning Representations (ICLR), 2021.}
{{\vspace{-0.5\baselineskip}}}

\ProjectEntry
{Overparameterisation and worst-case generalisation: friend or foe?}
{Aditya Krishna Menon, Ankit Singh Rawat, and Sanjiv Kumar.}
{In International Conference on Learning Representations (ICLR), 2021.}
{{\vspace{-0.5\baselineskip}}}

\ProjectEntry
{Long-tail learning via logit adjustment.}
{Aditya Krishna Menon, Sadeep Jayasumana, Ankit Singh Rawat, Himanshu Jain, Andreas Veit, and Sanjiv Kumar.}
{In International Conference on Learning Representations (ICLR), 2021.}
{{\vspace{-0.5\baselineskip}}}

\ProjectEntry
{Robust large-margin Learning in hyperbolic space.}
{Melanie Weber, Manzil Zaheer, Ankit Singh Rawat, Aditya Krishna Menon, and Sanjiv Kumar.}
{In Advances in Neural Information Processing Systems (NeurIPS), 2020.}
{{\vspace{-0.5\baselineskip}}}

\ProjectEntry
{Semantic label smoothing for sequence to sequence problems.}
{Michal Lukasik, Himanshu Jain, Aditya Krishna Menon, Seungyeon Kim, Srinadh Bhojanapalli, Felix Yu and Sanjiv Kumar.}
{In Empirical Methods in Natural Language Processing (EMNLP), 2020.}
{{\vspace{-0.5\baselineskip}}}

\ProjectEntry
{SupMMD: a sentence importance model for extractive summarization using maximum mean discrepancy.}
{Umanga Bista, Alexander Patrick Mathews, Aditya Krishna Menon, and Lexing Xie.}
{In Empirical Methods in Natural Language Processing Findings (EMNLP Findings), 2020.}
{{\vspace{-0.5\baselineskip}}}

\ProjectEntry
{Does label smoothing mitigate label noise?}
{Michal Lukasik, Srinadh Bhojanapalli, Aditya Krishna Menon, and Sanjiv Kumar.}
{In International Conference on Machine Learning (ICML), 2020.}
{{\vspace{-0.5\baselineskip}}}

\ProjectEntry
{Supervised learning: no loss no cry.}
{Richard Nock and Aditya Krishna Menon.}
{In International Conference on Machine Learning (ICML), 2020.}
{{\vspace{-0.5\baselineskip}}}

\ProjectEntry
{Federated learning with only positive labels.}
{Felix X. Yu, Ankit Singh Rawat, Aditya Krishna Menon, and Sanjiv Kumar.}
{In International Conference on Machine Learning (ICML), 2020.}
{{\vspace{-0.5\baselineskip}}}

\ProjectEntry
{Can gradient clipping mitigate label noise?}
{Aditya Krishna Menon, Ankit Singh Rawat, Sashank J. Reddi, and Sanjiv Kumar.}
{In International Conference on Learning Representations (ICLR), Addis Ababa 2020.}
{{\vspace{-0.5\baselineskip}}}

\ProjectEntry
{Noise-tolerant fair classification.}
{Alexandre Louis Lamy, Ziyuan Zhong, Aditya Krishna Menon, and Nakul Verma.}
{In Advances in Neural Information Processing Systems (NeurIPS), Vancouver, 2019.}
{{\vspace{-0.5\baselineskip}}}

\ProjectEntry
{Multilabel reductions: what is my loss optimising?}
{Aditya Krishna Menon, Ankit Singh Rawat, Sashank J. Reddi, and Sanjiv Kumar.}
{In Advances in Neural Information Processing Systems (NeurIPS), Vancouver, 2019.}
{{\vspace{-0.5\baselineskip}}}

\ProjectEntry
{Fairness risk measures.}
{Robert C. Williamson and Aditya Krishna Menon.}
{In International Conference on Machine Learning (ICML), Long Beach, 2019.}
{{\vspace{-0.5\baselineskip}}}

\ProjectEntry
{Complementary-label learning for arbitrary losses and models.}
{Takashi Ishida, Gang Niu, Aditya Krishna Menon, and Masashi Sugiyama.}
{In International Conference on Machine Learning (ICML), Long Beach, 2019.}
{{\vspace{-0.5\baselineskip}}}

\ProjectEntry
{Monge blunts Bayes: hardness results for adversarial training.}
{Zac Cranko, Aditya Krishna Menon, Richard Nock, Cheng-Soon Ong, Zhan Shi, and Christian Walder.}
{In International Conference on Machine Learning (ICML), Long Beach, 2019.}
{{\vspace{-0.5\baselineskip}}}

\ProjectEntry
{On the minimal supervision for training any binary classifier from only unlabeled data.}
{Nan Lu, Gang Niu, Aditya Krishna Menon and Masashi Sugiyama.}
{In International Conference on Learning Representations (ICLR), New Orleans, 2019.}
{{\vspace{-0.5\baselineskip}}}

\ProjectEntry
{Comparative document collection via classification.}
{Umanga Bista, Alexander Mathews, Minjeong Shin, Aditya Krishna Menon and Lexing Xie.}
{In AAAI Conference on Artificial Intelligence (AAAI), Honolulu, 2019.}
{{\vspace{-0.5\baselineskip}}}

\ProjectEntry
{The risk of trivial solutions in bipartite top ranking}
{Aditya Krishna Menon.}
{In Machine Learning, Volume 108, Issue 4, 2019.}
{{\vspace{-0.5\baselineskip}}}

\ProjectEntry
{Learning from binary labels with instance-dependent corruption. }
{Aditya Krishna Menon, Brendan van Rooyen, and Nagarajan Natarajan.}
{In Machine Learning, 2018.}
{{\vspace{-0.5\baselineskip}}}

\ProjectEntry
{The cost of fairness in binary classification. }
{Aditya Krishna Menon and Robert C. Williamson.}
{In Conference on Fairness, Accountability, and Transparency (FAT), New York City, 2018.}
{{\vspace{-0.5\baselineskip}}}

\ProjectEntry
{Proper losses for nonlinear Hawkes processes. }
{Aditya Krishna Menon and Young Lee.}
{In AAAI Conference on Artificial Intelligence (AAAI), New Orleans, 2018.}
{{\vspace{-0.5\baselineskip}}}

\ProjectEntry
{f-GANs in an information geometric nutshell. }
{Richard Nock, Zac Cranko, Aditya Krishna Menon, Lizhen Qu and Robert C. Williamson.}
{In Advances in Neural Information Processing Systems (NIPS), Los Angeles, 2017.}
{{\vspace{-0.5\baselineskip}}}

\ProjectEntry
{Predicting short-term public transport demand via inhomogeneous Poisson processes. }
{Aditya Krishna Menon and Young Lee.}
{In International Conference on Information and Knowledge Management (CIKM), Singapore, 2017.}
{{\vspace{-0.5\baselineskip}}}

\ProjectEntry
{Revisiting revisits in trajectory recommendation. }
{Aditya Krishna Menon, Dawei Chen, Lexing Xie and Cheng Soon Ong.}
{In RecSys Workshop on Recommender Systems for Citizens (CitRec), Como, 2017.}
{{\vspace{-0.5\baselineskip}}}

\ProjectEntry
{Robust, deep and inductive anomaly detection. }
{Raghavendra Chalapathy, Aditya Krishna Menon and Sanjay Chawla.}
{In European Conference on Machine Learning (ECML/PKDD), Skopje, 2017.}
{{\vspace{-0.5\baselineskip}}}

\ProjectEntry
{Making deep neural networks robust to label noise: a loss correction approach. }
{Giorgio Patrini, Alessandro Rozza, Aditya Krishna Menon, Richard Nock, Lizhen Qu.}
{In Computer Vision and Pattern Recognition (CVPR), Honolulu, 2017.}
{{\vspace{-0.5\baselineskip}}}

\ProjectEntry
{Low-rank linear cold-start recommendation from social data. }
{Suvash Sedhain, Aditya Krishna Menon, Scott Sanner, Lexing Xie, and Darius Braziunas.}
{In AAAI Conference on Artificial Intelligence (AAAI), San Francisco, 2017.}
{{\vspace{-0.5\baselineskip}}}

\ProjectEntry
{Bipartite ranking: a risk-theoretic perspective. }
{Aditya Krishna Menon and Robert C. Williamson.}
{In Journal of Machine Learning Research (JMLR), Volume 17, Issue 195. 2016.}
{{\vspace{-0.5\baselineskip}}}

\ProjectEntry
{A scaled Bregman theorem with applications. }
{Richard Nock, Aditya Krishna Menon and Cheng Soon Ong.}
{In Advances in Neural Information Processing Systems (NIPS), Barcelona, 2016.}
{{\vspace{-0.5\baselineskip}}}

\ProjectEntry
{Linking losses for density ratio and class-probability estimation. }
{Aditya Krishna Menon and Cheng Soon Ong.}
{In International Conference on Machine Learning (ICML), New York City, 2016.}
{{\vspace{-0.5\baselineskip}}}

\ProjectEntry
{Practical linear models for large-scale one-class collaborative filtering.}
{Suvash Sedhain, Hung Bui, Jaya Kawale, Nikos Vlassis, Branislav Kveton, Aditya Krishna Menon, Trung Bui and Scott Sanner.}
{In International Joint Conference on Artificial Intelligence (IJCAI), New York City, 2016.}
{{\vspace{-0.5\baselineskip}}}

\ProjectEntry
{On the effectiveness of linear models for one-class collaborative filtering.}
{Suvash Sedhain, Aditya Krishna Menon, Scott Sanner and Darius Braziunas.}
{In AAAI Conference on Artificial Intelligence (AAAI), Phoenix, 2016.}
{{\vspace{-0.5\baselineskip}}}

\ProjectEntry
{Learning with symmetric label noise: the importance of being unhinged.}
{Brendan van Rooyen, Aditya Krishna Menon and Robert C. Williamson.}
{In Advances in Neural Processing Systems (NIPS), Montreal, 2015.}
{{\vspace{-0.5\baselineskip}}}

\ProjectEntry
{Fine-grained OD estimation with automated zoning and sparsity regularisation.}
{Aditya Krishna Menon, Chen Cai, Weihong Wang, Tao Wen and Fang Chen.}
{In Transportation Research Part B: Methodological, Volume 80, October 2015, Pages 150-172.}
{{\vspace{-0.5\baselineskip}}}

\ProjectEntry
{Learning from corrupted binary labels via class-probability estimation.}
{Aditya Krishna Menon, Brendan van Rooyen, Cheng Soon Ong and Robert C. Williamson.}
{In International Conference on Machine Learning (ICML), Lille, 2015.}
{{\vspace{-0.5\baselineskip}}}

\ProjectEntry
{AutoRec: autoencoders meet collaborative filtering.}
{Suvash Sedhain, Aditya Krishna Menon, Scott Sanner and Lexing Xie.}
{In International World Wide Web Conference (WWW), Florence, 2015.}
{{\vspace{-0.5\baselineskip}}}

\ProjectEntry
{Cross-modal retrieval: a pairwise classification approach.}
{Aditya Krishna Menon, Didi Surian and Sanjay Chawla.}
{In SIAM Conference on Data Mining (SDM), Vancouver, 2015.}
{{\vspace{-0.5\baselineskip}}}

\ProjectEntry
{An approach to sparse, fine-grained OD estimation.}
{Aditya Krishna Menon, Chen Cai, Weihong Wang, Tao Wen and Fang Chen.}
{In 94th Annual Meeting of the Transporation Research Board (TRB), Washington DC, 2015.}
{{\vspace{-0.5\baselineskip}}}

\ProjectEntry
{Bayes-optimal scorers for bipartite ranking.}
{Aditya Krishna Menon and Robert C. Williamson.}
{In Conference on Learning Theory (COLT), Barcelona, 2014.}
{{\vspace{-0.5\baselineskip}}}

\ProjectEntry
{Inappropriate access detection for electronic health records using collaborative filtering.}
{Aditya Krishna Menon, Xiaoqian Jiang, Jihoon Kim, Lucila Ohno-Machado, and Jaideep Vaidya.}
{In Machine Learning, Volume 95 Number 1, Special Issue on Machine Learning for Society, 2014.}
{{\vspace{-0.5\baselineskip}}}

\ProjectEntry
{A colorful approach to text processing by example.}
{Kuat Yessenov, Shubham Tulsiani, Aditya Krishna Menon, Robert C. Miller, Sumit Gulwani, Butler Lampson, and Adam Kalai.}
{In ACM Symposium on User Interface Software and Technology (UIST), 2013.}
{{\vspace{-0.5\baselineskip}}}

\ProjectEntry
{Beam search algorithms for multilabel learning.}
{Abhishek Kumar, Shankar Vembu, Aditya Krishna Menon, and Charles Elkan.}
{In Machine Learning, 2013.}
{{\vspace{-0.5\baselineskip}}}

\ProjectEntry
{On the statistical consistency of algorithms for binary classification under class imbalance.}
{Aditya Krishna Menon, Harikrishna Narasimhan, Shivani Agarwal and Sanjay Chawla.}
{In International Conference on Machine Learning (ICML), Atlanta, 2013.}
{{\vspace{-0.5\baselineskip}}}

\ProjectEntry
{A machine learning framework for programming by example.}
{Aditya Krishna Menon, Omer Tamuz, Sumit Gulwani, Butler Lampson, and Adam Tauman Kalai.}
{In International Conference on Machine Learning (ICML), Atlanta, 2013.}
{{\vspace{-0.5\baselineskip}}}

\ProjectEntry
{Learning and inference in Probabilistic Classifier Chains with beam search.}
{Abhishek Kumar, Shankar Vembu, Aditya Krishna Menon, and Charles Elkan.}
{In Machine Learning and Knowledge Discovery in Databases - European Conference (ECML-PKDD), Proceedings Part I, 2012.}
{{\vspace{-0.5\baselineskip}}}

\ProjectEntry
{Doubly optimized calibrated Support Vector Machine (DOC-SVM): an algorithm for joint optimization of discrimination and calibration.}
{Xiaoqian Jiang, Aditya Krishna Menon, Shuang Wang, Jihoon Kim, and Lucila Ohno-Machado.}
{In PLoS ONE, 7(11): e48823, 2012.}
{{\vspace{-0.5\baselineskip}}}

\ProjectEntry
{Predicting accurate probabilities with a ranking loss.}
{Aditya Krishna Menon, Xiaoqian Jiang, Shankar Vembu, Charles Elkan, and Lucila Ohno-Machado.}
{In International Conference on Machine Learning (ICML), Edinburgh, 2012.}
{{\vspace{-0.5\baselineskip}}}

\ProjectEntry
{Link prediction via matrix factorization.}
{Aditya Krishna Menon, Charles Elkan.}
{In Machine Learning and Knowledge Discovery In Databases - European Conference, ECML-PKDD, Proceedings Part II, 2011.}
{{\vspace{-0.5\baselineskip}}}

\ProjectEntry
{Response prediction using collaborative filtering with hierarchies and side-information.}
{Aditya Krishna Menon, Krishna-Prasad Chitrapura, Sachin Garg, Deepak Agarwal, and Nagaraj Kota.}
{In Knowledge Discovery and Data Mining (KDD), San Diego, 2011.}
{{\vspace{-0.5\baselineskip}}}

\ProjectEntry
{Fast algorithms for approximating the singular value decomposition.}
{Aditya Krishna Menon, Charles Elkan.}
{In Transactions of Knowledge and Data Discovery: Special Issue on Large-Scale Data Mining (TKDD-LDMTA), 2010.}
{{\vspace{-0.5\baselineskip}}}

\ProjectEntry
{A log-linear model with latent features for dyadic prediction.}
{Aditya Krishna Menon, Charles Elkan.}
{In International Conference on Data Mining (ICDM), Sydney, 2010.}
{{\vspace{-0.5\baselineskip}}}

\ProjectEntry
{Predicting labels for dyadic data.}
{Aditya Krishna Menon, Charles Elkan.}
{In Data Mining and Knowledge Discovery: Special Issue on Papers from ECML-PKDD, Volume 21, Number 2, 2010.}
{{\vspace{-0.5\baselineskip}}}

\ProjectEntry
{An incremental data-stream sketch using sparse random projections.}
{Aditya Krishna Menon, Gia Vinh Anh Pham, Sanjay Chawla and Anastasios Viglas.}
{In SIAM Conference on Data Mining (SDM), Minnesota, 2007.}


\vskip0.5\baselineskip

\end{document}
